\documentclass[10pt,a4paper]{article}
\usepackage[utf8]{inputenc}
\usepackage[T1]{fontenc}
\usepackage{amsmath}
\usepackage{amssymb}
\usepackage[scale=0.75]{geometry}
\usepackage{graphicx}
\usepackage{multicol, multirow}
\usepackage{booktabs}
\usepackage{rotating}
\usepackage{setspace}
\title{CSC6780 - Data Science; Assignment 8}
\author{Jesse Annan \hspace{0.5cm} $\|$ \hspace{0.5cm} ID: 002708111}
\newcommand{\m}{\mathbb{M}}
\newcommand{\mm}{\textbf{M}}
\newcommand{\h}{\textbf{H}}
\newcommand{\f}{\textbf{F}}
\newcommand{\vv}{\textbf{V}}
\newcommand{\cc}{\textbf{C}}
\newcommand{\p}{\textbf{P}}
\newcommand{\g}{\textbf{G}}
\newcommand{\ag}{\textbf{A}}
\newcommand{\pt}{\textbf{PT}}
\newcommand{\pc}{\textbf{PC}}
\begin{document}
	\maketitle
	
	\clearpage
	
	\textbf{Question 1} \newline \\ \textit{Find data in the attached notebook.}
		\begin{enumerate}
			\item[(i)] The sums of squared errors
				\begin{equation*}
					\begin{split}
						\textbf{Error[Model 1 Prediction]} & = \frac{1}{2} \sum_{1}^{30} (Target[i] - Model1 Prediction[i]) ^ 2 \\
						& = 16750 \\
						\textbf{Error[Model 2 Prediction]} & = \frac{1}{2} \sum_{1}^{30} (Target[i] - Model2 Prediction[i]) ^ 2 \\
						& = 47369 \\
					\end{split}
				\end{equation*}
			
			\item[(ii)] The $R^2$ measure
				\begin{equation*}
					\begin{split}
						\textbf{$R^2$ [Model 1 Prediction]} & = 1 - \frac{ \frac{1}{2} \sum_{1}^{30} (Target[i] - Model1 Prediction[i]) ^ 2 }{ \frac{1}{2} \sum_{1}^{30} (Target[i] - Target\_mean) ^ 2 }  \\
						& \approx 0.92844 \\
						\textbf{$R^2$ [Model 2 Prediction]} & = 1 - \frac{ \frac{1}{2} \sum_{1}^{30} (Target[i] - Model2 Prediction[i]) ^ 2 }{ \frac{1}{2} \sum_{1}^{30} (Target[i] - Target\_mean) ^ 2 }  \\
						& \approx 0.79762
					\end{split}
				\end{equation*}
		\end{enumerate}
	
	\newpage
	
	\textbf{Question 2} \newline \\ Model: \newline
	\texttt{HEATING LOAD = -26.030 + 0.0497 x SURFACE AREA + 4.942 x HEIGHT - 0.090 x ROOF AREA + 20.523 \hspace*{2.7cm} x GLAZING AREA}
	\begin{equation*}
		\begin{split}
			\texttt{ID1: Heating Load} & = - 26.030 + 0.0497 * 784.0 + 4.942 * 3.5 - 0.090 * 220.5 + 20.523 * 0.25 \\
			& 15.51755 \\
			\texttt{ID2: Heating Load} & = - 26.030 + 0.0497 * 710.5 + 4.942 * 3.0 - 0.090 * 210.5 + 20.523 * 0.10 \\
			& 7.21515 \\
			\texttt{ID3: Heating Load} & = - 26.030 + 0.0497 * 563.5 + 4.942 * 7.0 - 0.090 * 122.5 + 20.523 * 0.40 \\
			& 33.75415 \\
			\texttt{ID4: Heating Load} & = - 26.030 + 0.0497 * 637.0 + 4.942 * 6.0 - 0.090 * 147.0 + 20.523 * 0.60 \\
			& 34.3647 \\
		\end{split}
	\end{equation*}
	
	\newpage
	
	\textbf{Question 3} \newline \\ \texttt{Imputation} \newline
		age[3] = 32.7 \hspace{1cm} and \hspace{1cm} socio\_economic\_band[5] = 'a' \newline \\
		\texttt{Normalized Features}
		\begin{table}[h!]
			\begin{tabular}{ccccc}
			\toprule
			\textbf{ID} & \textbf{AGE} & \textbf{SOCIO ECONOMIC BAND} & \textbf{SHOP FREQUENCY} & \textbf{SHOP VALUE} \\ \midrule
			1 & -0.11111 & a & -0.34615 & 0.42127 \\
			2 & 0.68889 & b & -0.46154 & -0.07559 \\
			3 & -1.00000 & c & 1.23077 & -0.95490 \\
			4 & -0.34667 & b & -0.56538 & 0.76890 \\
			5 & 0.95556 & a & -0.75000 & -0.06513 \\ 
			\bottomrule 
		\end{tabular}
		\end{table}
		
		\begin{equation*}
			\begin{split}
				\texttt{Logistic(X)} & = \frac{1}{1 + \exp{(-X)}} \\
				ID1\_wd & = 0.6679 - 0.5795 * -0.11111 + 0 + 2.0499 * -0.34615 + 3.4091 * 0.42127 \\
				ID1 & = \texttt{Logistic(ID1\_wd)} \approx 0.811357 \\
				ID2\_wd & = 0.6679 - 0.5795 * 0.68889 - 0.1981 + 2.0499 * -0.46154 + 3.4091 * -0.07559 \\
				ID2 & = \texttt{Logistic(ID2\_wd)} \approx 0.243571 \\
				ID3\_wd & = 0.6679 - 0.5795 - 1 - 0.2318 + 2.0499 * 1.23077 + 3.4091 * -0.95490 \\
				ID3 & = \texttt{Logistic(ID3\_wd)} \approx 0.570335 \\
				ID4\_wd & = 0.6679 - 0.5795 * -0.34667 - 0.1981 + 2.0499 * -0.56538 + 3.4091 * 0.76890 \\
				ID4 & = \texttt{Logistic(ID4\_wd)} \approx 0.894065 \\
				ID5\_wd & = 0.6679 - 0.5795 * 0.95556 + 0 + 2.0499 * -0.75000 + 3.4091 * -0.06513 \\
				ID5 & = \texttt{Logistic(ID5\_wd)} \approx 0.161744
			\end{split}
		\end{equation*}
	
	\newpage
	
	\textbf{Question 4} 
		\begin{enumerate}
			\item[(a)] Yes,  with a reasonable choice of k. Similarity-based predictive modeling approach (KNN) will be a good choice for this data set.
			
			\item[(b)] \begin{equation*}
				\begin{split}
					\texttt{Logistic(X)} & = \frac{1}{1 + \exp{(-X)}} \\
					ID1\_wd & = -0.848 * (1) + 1.545 * (0.50) - 1.942 * (0.75) + 1.973 * (0.50^2) \\
					& + 2.495 * (0.75^2) + 0.104 * (0.50^3) + 0.095 * (0.75^3) + 3.009 * (0.50 * 0.75) \\
					ID1 & = \texttt{Logistic(ID1\_wd)} \approx 0.824356 \\
					ID2\_wd & = -0.848 * (1) + 1.545 * (0.10) - 1.942 * (0.75) + 1.973 * (0.10^2) \\
					& + 2.495 * (0.75^2) + 0.104 * (0.10^3) + 0.095 * (0.75^3) + 3.009 * (0.10 * 0.75) \\
					ID2 & = \texttt{Logistic(ID1\_wd)} \approx 0.386754 \\
					ID3\_wd & = -0.848 * (1) + 1.545 * (-0.47) - 1.942 * (-0.39) + 1.973 * (-0.47^2) \\
					& + 2.495 * (-0.39^2) + 0.104 * (-0.47^3) + 0.095 * (-0.39^3) + 3.009 * (-0.47 * -0.39) \\
					ID3 & = \texttt{Logistic(ID1\_wd)} \approx 0.630339 \\
					ID4\_wd & = -0.848 * (1) + 1.545 * (-0.47) - 1.942 * (0.18) + 1.973 * (-0.47^2) \\
					& + 2.495 * (0.18^2) + 0.104 * (-0.47^3) + 0.095 * (0.18^3) + 3.009 * (-0.47 * 0.18) \\
					ID4 & = \texttt{Logistic(ID1\_wd)} \approx 0.158179 \\
				\end{split}
			\end{equation*}
		\end{enumerate}
		
	
		
	
\end{document}