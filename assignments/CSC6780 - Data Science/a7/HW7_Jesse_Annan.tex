\documentclass[10pt,a4paper]{article}
\usepackage[utf8]{inputenc}
\usepackage[T1]{fontenc}
\usepackage{amsmath}
\usepackage{amssymb}
\usepackage[scale=0.75]{geometry}
\usepackage{graphicx}
\usepackage{multicol, multirow}
\usepackage{booktabs}
\usepackage{rotating}
\usepackage{setspace}
\title{CSC6780 - Data Science; Assignemt 7}
\author{Jesse Annan \hspace{0.5cm} $\|$ \hspace{0.5cm} ID: 002708111}
\newcommand{\m}{\mathbb{M}}
\newcommand{\mm}{\textbf{M}}
\newcommand{\h}{\textbf{H}}
\newcommand{\f}{\textbf{F}}
\newcommand{\vv}{\textbf{V}}
\newcommand{\cc}{\textbf{C}}
\newcommand{\p}{\textbf{P}}
\newcommand{\g}{\textbf{G}}
\newcommand{\ag}{\textbf{A}}
\newcommand{\pt}{\textbf{PT}}
\newcommand{\pc}{\textbf{PC}}
\begin{document}
	\maketitle
	
	\clearpage
	
	\textbf{Question 1} $$\p(event) = \frac{\#coin flips \ \ \cc ombination \ \ \#exact heads}{2^{\#coin flips}}$$
	
	\begin{enumerate}
		\item[(a)] Let A:= the probability that exactly two of them will get heads. \newline
		$$\p(A) = \frac{3 \cc 2}{2^{3}} = \frac{3}{8}$$ 
		
		\item[(b)] Let B:= the probability that exactly eight eight of them will get heads. \newline
		$$\p(B) = \frac{20 \cc 8}{2^{20}} \approx 0.12013435 $$
		
		\item[(c)] Let C:= the probability that at least four of them will get heads. %\newline
		\begin{equation*}
			\begin{split}
				\p(C) & = 1 - \p(\leq \text{3 people getting heads}) \\
				& = 1 - \left[ \frac{20 \cc 0}{2^{20}} + \frac{20 \cc 1}{2^{20}} + \frac{20 \cc 2}{2^{20}} + \frac{20 \cc 3}{2^{20}} \right] \\
				\p(C) & \approx 0.99871159
			\end{split}
		\end{equation*}
	\end{enumerate}
	
	\newpage
	
	\textbf{Question 2} \begin{table}[h!]
		\begin{tabular}{ccccc}
			\toprule
			 \textbf{ID} & \textbf{HEADACHE} & \textbf{FEVER} & \textbf{VOMITING} & \textbf{MENINGITIS} \\ \midrule
			1 & true & true & false & false \\
			2 & false & true & false & false \\
			3 & true & false & true & false \\
			4 & true & false & true & false \\
			5 & false & true & false & true \\
			6 & true & false & true & false \\
			7 & true & false & true & false \\
			8 & true & false & true & true \\
			9 & false & true & false & false \\ 
			10 & true & false & true & true  \\
			\bottomrule
		\end{tabular}
		\end{table}
	
		Let \h := HEADACHE, \f := FEVER, \vv := VOMITING, \newline $\m$ := MENINGITIS, $\mm_{T}$ := MENINGITIS = True, $\mm_{F}$ := MENINGITIS = False
	
		\begin{enumerate}
			\item[(a)] $\p(\text{\vv = True}) = \frac{6}{10}$
			
			\item[(b)] $\p(\text{\h = True}) = \frac{3}{10}$
			
			\item[(c)] \begin{equation*}
				\begin{split}
					\p(\text{\h = True,\; \vv = False}) & = \p(\text{\vv = False | \h = True}) \\
					& \times \p(\text{\h = True}) \\
					& = \frac{1}{7} \times \frac{7}{10} \\
					\p(\text{\h = True,\; \vv = False}) & = \frac{1}{10}
				\end{split}
			\end{equation*}
			
			\item[(d)] $\p(\text{\vv = False | \h = True}) = \frac{1}{7}$
			
			\item[(e)] $\p(\m\text{ | \f = True, \vv = False}) = (\p(\text{$\mm_{T}$ | \f = True, \vv = False}) \;,\; \p(\text{$\mm_{F}$ | \f = True, \vv = False}))$ %\newline
				\begin{equation*}
					\begin{split}
						\p(\text{$\mm_{T}$ | \f = True, \vv = False}) & = \frac{ \p(\text{\f = True, \vv = False | $\mm_{T}$}) \times \p(\mm_{T}) }{\p(\text{\f = True, \vv = False})} \\
						& = \frac{ \p(\text{\f = True | $\mm_{T}$}) \times \p(\text{\vv = False | \f = True, $\mm_{T}$}) \times \p(\mm_{T}) }{\p(\text{\vv = False | \f = True}) \times \p(\text{\f = True})} \\
						& = \frac{1/3 \times 1 \times 3/10}{1 \times 4/10} \\
						\p(\text{$\mm_{T}$ | \f = True, \vv = False}) & = 0.25 \\
						\p(\text{$\mm_{F}$ | \f = True, \vv = False}) & = \frac{ \p(\text{\f = True, \vv = False | $\mm_{F}$}) \times \p(\mm_{F}) }{\p(\text{\f = True, \vv = False})} \\
						& = \frac{ \p(\text{\f = True | $\mm_{F}$}) \times \p(\text{\vv = False | \f = True, $\mm_{F}$}) \times \p(\mm_{F}) }{\p(\text{\vv = False | \f = True}) \times \p(\text{\f = True})} \\
						& = \frac{3/7 \times 1 \times 7/10}{1 \times 4/10} \\
						\p(\text{$\mm_{F}$ | \f = True, \vv = False}) & = 0.75
					\end{split}
				\end{equation*}
				$\p(\m\text{ | \f = True, \vv = False}) = (0.25 \;,\; 0.75)$
				
		\end{enumerate}
	
	\newpage
	
	\textbf{Question 3} \begin{table}[h!]
		\begin{tabular}{cccccc}
			\toprule
			\textbf{ID} & \textbf{OCCUPATION} & \textbf{GENDER} & \textbf{AGE} & \textbf{POLICY TYPE} & \textbf{PREF CHANNEL} \\ \midrule
			1 & lab tech & female & 43 & planC & email \\
			2 & farmhand & female & 57 & planA & phone \\
			3 & biophysicist & male & 21 & planA & email \\ 
			4 & sheriff & female & 47 & planB & phone \\
			5 & painter & male & 55 & planC & phone \\
			6 & manager & male & 19 & planA & email \\
			7 & geologist & male & 49 & planC & phone \\
			8 & messenger & male & 51 & planB & email \\
			9 & nurse & female & 18 & planC & phone \\ \bottomrule 
		\end{tabular}
	\end{table}
	
	\begin{enumerate}
		\item[(a)] $Bins = 3$ \newline
			\textit{young} : \{18,\; 19,\; 21\} \hspace{1cm} \textit{middle-aged} : \{43,\; 47,\; 49\} \hspace{1cm} \textit{mature} : \{51,\; 55,\; 57\} \newline
			Defining the range of the \textbf{AGE} column by averaging the ends from the above three levels:
			\begin{table}[h]
				\begin{tabular}{ccccc}
					& & young & $\leq$ & 32 \\
					32 & $<$ & middle-age & $\leq$ & 50 \\
					50 & $<$ & mature & &
				\end{tabular}
			\end{table}
			\begin{table}[h!]
				\begin{tabular}{cccccc}
					\toprule
					\textbf{ID} & \textbf{OCCUPATION} & \textbf{GENDER} & \textbf{AGE} & \textbf{POLICY TYPE} & \textbf{PREF CHANNEL} \\ \midrule
					1 & lab tech & female & middle-aged & planC & email \\
					2 & farmhand & female & mature & planA & phone \\
					3 & biophysicist & male & young & planA & email \\ 
					4 & sheriff & female & middle-aged & planB & phone \\
					5 & painter & male & mature & planC & phone \\
					6 & manager & male & young & planA & email \\
					7 & geologist & male & middle-aged & planC & phone \\
					8 & messenger & male & mature & planB & email \\
					9 & nurse & female & young & planC & phone \\ \bottomrule 
				\end{tabular}
			\end{table}
		
		\item[(b)] Take out \textbf{"ID" and "OCCUPATION"} \newline
			This is because both ID and OCCUPATION have too many unique levels which will create a high dimensionality problem for any model.
		
		
		\newpage
		\item[(c)] Let \g := GENDER, \ag := AGE, \pt := POLICY TYPE, \pc := PREF CHANNEL
			\begin{table}[h!]
				\begin{tabular}{l|r}
					\toprule
					\p(\pc = email) = $\frac{4}{9}$ & \p(\pc = phone) = $\frac{5}{9}$ \\
					\p(\pt = planA | \pc = email) = $\frac{2}{4}$ & \p(\pt = planA | \pc = phone) = $\frac{1}{5}$ \\
					\p(\pt = planB | \pc = email) = $\frac{1}{4}$ & \p(\pt = planB | \pc = phone) = $\frac{1}{5}$ \\
					\p(\pt = planC | \pc = email) = $\frac{1}{4}$ & \p(\pt = planC | \pc = phone) = $\frac{3}{5}$ \\
					\p(\g = male | \pc = email) = $\frac{3}{4}$ & \p(\g = male | \pc = phone) = $\frac{2}{5}$ \\
					\p(\g = female | \pc = email) = $\frac{1}{4}$ & \p(\g = female | \pc = phone) = $\frac{3}{5}$ \\
					\p(\ag = young | \pc = email) = $\frac{2}{4}$ & \p(\ag = young | \pc = phone) = $\frac{1}{5}$ \\
					\p(\ag = middle-age | \pc = email) = $\frac{1}{4}$ & \p(\ag = middle-age | \pc = phone) = $\frac{2}{5}$ \\
					\p(\ag = mature | \pc = email) = $\frac{1}{4}$ & \p(\ag = mature | \pc = phone) = $\frac{2}{5}$ \\
					\bottomrule
				\end{tabular}
			\end{table}
		
		\item[(d)] \texttt{Query: GENDER = female, AGE = 30, POLICY = planA}
		\begin{table}[h!]
			\begin{tabular}{l|r}
				\toprule
				\p(\pc = email) = $\frac{4}{9}$ & \p(\pc = phone) = $\frac{5}{9}$ \\
				\p(\pt = planA | \pc = email) = $\frac{2}{4}$ & \p(\pt = planA | \pc = phone) = $\frac{1}{5}$ \\
				\p(\g = female | \pc = email) = $\frac{1}{4}$ & \p(\g = female | \pc = phone) = $\frac{3}{5}$ \\
				\p(\ag = young | \pc = email) = $\frac{2}{4}$ & \p(\ag = young | \pc = phone) = $\frac{1}{5}$ \\
				\bottomrule
			\end{tabular}
		\end{table}
	
		\p(\pc = email | \pt = planA, \g = female, \ag = young) = $ \frac{1}{4} \times \frac{2}{4} \times \frac{2}{4} \times \frac{4}{9} = \frac{1}{36} \approx 0.0277\dot{7} $ \newline
		\p(\pc = phone | \pt = planA, \g = female, \ag = young) = $ \frac{3}{5} \times \frac{1}{5} \times \frac{1}{5} \times \frac{5}{9} = \frac{1}{75} \approx 0.0133\dot{3} $ \newline \newline
		\texttt{Query Prediction: "email"}
	
	\end{enumerate}
	
	\newpage
	
	\textbf{Question 4} size of entertainment = 700 \hspace{1cm} size of education = 300
		\begin{table}[h!]
			\begin{tabular}{cccccc}
				\toprule
				\multicolumn{6}{c}{Word-document counts for the entertainment dataset:} \\ \midrule
				fun & is & machine & christmas & family & learning \\
				415 & 695 & 35 & 0 & 400 & 70 \\
				\midrule
				\multicolumn{6}{c}{Word-document counts for the education dataset:} \\ \midrule
				fun & is & machine & christmas & family & learning \\
				200 & 295 & 120 & 0 & 10 & 105 \\
				\bottomrule
			\end{tabular}
		\end{table}
		
		\begin{enumerate}
			\item[(a)] \texttt{Query: "machine learning is fun"}
				\begin{table}[h!]
					\begin{tabular}{lcc}
						\p(entertainment) & = & $\frac{700}{1000}$ \\
						\p(machine | entertainment) & = & $\frac{35}{700}$ \\
						\p(learning | entertainment) & = & $\frac{70}{700}$ \\
						\p(is | entertainment) & = & $\frac{695}{700}$ \\
						\p(fun | entertainment) & = & $\frac{415}{700}$ \\
						\p(education) & = & $\frac{300}{1000}$ \\
						\p(machine | education) & = & $\frac{120}{300}$ \\
						\p(learning | education) & = & $\frac{105}{300}$ \\
						\p(is | education) & = & $\frac{295}{300}$ \\
						\p(fun | education) & = & $\frac{200}{300}$ \\
					\end{tabular}
				\end{table}
			
				case 1: $\frac{700}{1000} \times \frac{35}{700} \times \frac{70}{700} \times \frac{695}{700} \times \frac{415}{700} \approx 0.00206179 $ \newline \newline
				case 2: $ \frac{300}{1000} \times \frac{120}{300} \times \frac{105}{300} \times \frac{295}{300} \times \frac{200}{300} \approx 0.02753333 $ \newline \newline
				\texttt{Query Prediction: "education"}
			
			\item[(b)] \texttt{Query: "christmas family fun"}
			\begin{table}[h!]
				\begin{tabular}{lcc}
					\p(entertainment) & = & $\frac{700}{1000}$ \\
					\p(christmas | entertainment) & = & $0$ \\
					\p(family | entertainment) & = & $\frac{400}{700}$ \\
					\p(fun | entertainment) & = & $\frac{415}{700}$ \\
					\p(education) & = & $\frac{300}{1000}$ \\
					\p(christmas | education) & = & $0$ \\
					\p(family | education) & = & $\frac{10}{300}$ \\
					\p(fun | education) & = & $\frac{200}{300}$ \\
				\end{tabular}
			\end{table}
			
			case 1: $\frac{700}{1000} \times 0 \times \frac{400}{700} \times \frac{415}{700} = 0 $ \newline \newline
			case 2: $ \frac{300}{1000} \times 0 \times \frac{10}{300} \times \frac{200}{300} = 0 $ \newline \newline
			\texttt{Query Prediction: N/A}
			
			\item[(c)]  
		\end{enumerate}
		
		
	
		
	
\end{document}