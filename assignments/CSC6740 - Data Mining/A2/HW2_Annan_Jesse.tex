\documentclass[12pt]{article}
\usepackage[utf8]{inputenc}
\usepackage[T1]{fontenc}
\usepackage{amsmath}
\usepackage{amsfonts}
\usepackage{amssymb}
\usepackage{graphicx}
\usepackage{enumerate}

\title{Data Mining; Assignemt 2}
\author{Name: Jesse Annan \hspace{0.5cm} | \hspace{0.5cm} ID: 002708111}
\date{October 2022}

\begin{document}

\maketitle

\clearpage

\begin{enumerate}
    \item[Example 1:] "kitten" $\longrightarrow$ "sitting" \newline
    Output: 3 \newline
    \begin{tabular}{|c|c|c|c|c|c|c|c|c|}
        \hline
          &   & S & I & T & T & I & N & G \\ \hline
          & 0 & 1 & 2 & 3 & 4 & 5 & 6 & 7 \\ \hline
        K & 1 & 1 & 2 & 3 & 4 & 5 & 6 & 7 \\ \hline
        I & 2 & 2 & 1 & 2 & 3 & 4 & 5 & 6 \\ \hline
        T & 3 & 3 & 2 & 1 & 2 & 3 & 4 & 5 \\ \hline
        T & 4 & 4 & 3 & 2 & 1 & 2 & 3 & 4 \\ \hline
        E & 5 & 5 & 4 & 3 & 2 & 2 & 3 & 4 \\ \hline
        N & 6 & 6 & 5 & 4 & 3 & 3 & 2 & \boxed{3} \\ \hline
    \end{tabular}

	\hfill

    \item[Example 2:] "GUMBO" $\longrightarrow$ "GAMBOL" \newline
    Output: 2 \newline
    \begin{tabular}{|c|c|c|c|c|c|c|c|}
         \hline
          &   & G & A & M & B & O & L \\ \hline
          & 0 & 1 & 2 & 3 & 4 & 5 & 6 \\ \hline
        G & 1 & 0 & 1 & 2 & 3 & 4 & 5 \\ \hline
        U & 2 & 1 & 1 & 2 & 3 & 4 & 5 \\ \hline
        M & 3 & 2 & 2 & 1 & 2 & 3 & 4 \\ \hline
        B & 4 & 3 & 3 & 2 & 1 & 2 & 3 \\ \hline
        O & 5 & 4 & 4 & 3 & 2 & 1 & \boxed{2} \\ \hline
    \end{tabular}
\end{enumerate}


Yes the output result makes sense. I realized that the cell we are trying to compute is almost always the min of the three closest (computed) cells. 

\end{document}