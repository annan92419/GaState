\sloppypar

    \ac{erc} is an increasingly popular, yet unresolved, and challenging task in Natural Language Processing. Its aim is to identify the emotional states of speakers engaged in dialogue. \ac{erc} has various applications in human-computer interaction, social media analysis, mental health assessment, and affective computing. Solving the \ac{erc} task is a necessary step towards creating an empathetic \ac{adh}. An \ac{adh} is an entity finely attuned to emotions, enhancing its ability to engage users in a profoundly natural and empathetic manner. In this survey, we review recent advances in \ac{erc} methods, particularly focusing on multimodal approaches that utilize various information sources like text, audio, and visual cues. We categorize existing methods into two types: RNN-based and transformer-based, then discuss their respective advantages and disadvantages. Additionally, we compare their performance using two widely-used benchmark datasets: IEMOCAP (dyadic) and MELD (dialogue). Furthermore, we present the primary challenges and open issues in \ac{erc}, highlighting potential directions for future research. \newline \newline 
    \textit{\textbf{Keywords:} Natural Language Processing, Emotion Recognition, Autonomous Digital Human, Conversation Analysis, Dialogue Processing, Transformer, RNN}

    